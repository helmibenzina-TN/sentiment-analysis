\documentclass[a4paper,12pt]{report}
\usepackage[utf8]{inputenc}
\usepackage[T1]{fontenc}
\usepackage[french]{babel}
\usepackage{graphicx}
\usepackage{geometry}
\usepackage{hyperref}
\usepackage{titlesec}
\usepackage{fancyhdr}

\geometry{hmargin=2.5cm,vmargin=2.5cm}

% Configuration des liens
\hypersetup{
    colorlinks=true,
    linkcolor=black,
    filecolor=magenta,      
    urlcolor=blue,
}

% En-tête et pied de page
\pagestyle{fancy}
\fancyhf{}
\rhead{\thepage}
\lhead{Rapport de Stage - Analyse de Sentiments}

\title{\textbf{Rapport de Stage}\\Architecture et Développement d'une Application d'Analyse de Sentiments}
\author{Étudiant}
\date{Janvier 2026}

\begin{document}

\maketitle

\tableofcontents
\newpage

\chapter{Introduction}

\section{Contexte}
Le marché des smartphones est aujourd'hui saturé de références, rendant le choix difficile pour les consommateurs. Au-delà des fiches techniques, l'expérience utilisateur réelle, partagée massivement sur les réseaux sociaux comme Twitter (X), est un indicateur clé de la qualité d'un produit.

\section{Objectif du Stage}
L'objectif de ce stage a été de concevoir et développer une application web complète ("Full Stack") capable de :
\begin{itemize}
    \item Récupérer des avis utilisateurs (tweets).
    \item Analyser automatiquement le sentiment (positif, négatif, neutre) de ces avis grâce au Traitement du Langage Naturel (NLP).
    \item Restituer ces informations sous forme de tableaux de bord interactifs et comparatifs.
\end{itemize}

\chapter{Analyse des Besoins}

\section{Besoins Fonctionnels}
Le système a été conçu pour répondre aux besoins suivants :
\begin{enumerate}
    \item \textbf{Authentification} : Sécurisation de l'accès via inscription et connexion (JWT).
    \item \textbf{Recherche} : Saisie d'un nom de smartphone (ex: "iPhone 15") et lancement de l'analyse.
    \item \textbf{Visualisation} : Affichage des scores globaux et par catégorie (Batterie, Caméra, Prix).
    \item \textbf{Comparaison} : Possibilité de comparer les métriques de deux smartphones côte à côte.
    \item \textbf{Historique} : Sauvegarde des dernières recherches effectuées par l'utilisateur.
\end{enumerate}

\section{Besoins Non-Fonctionnels}
\begin{itemize}
    \item \textbf{Performance} : Utilisation d'un cache serveur pour éviter de réinterroger les API externes ou de relancer l'analyse NLP inutilement.
    \item \textbf{Ergonomie} : Interface utilisateur fluide et responsive développée avec Angular.
    \item \textbf{Scalabilité} : Architecture découplée (Frontend/Backend) facilitant les évolutions futures.
\end{itemize}

\section{Modélisation des Cas d'Utilisation}
Le diagramme ci-dessous illustre les principales interactions des acteurs avec le système.

\begin{figure}[h]
    \centering
    \includegraphics[width=0.9\textwidth]{rapport_resources/usecase.png}
    \caption{Diagramme de Cas d'Utilisation}
\end{figure}

\chapter{Conception et Modélisation}

\section{Diagramme d'Activité : Authentification}
Le processus d'authentification est critique pour la sécurité de l'application.

\begin{figure}[h]
    \centering
    \includegraphics[width=0.7\textwidth]{rapport_resources/activity.png}
    \caption{Processus d'Authentification}
\end{figure}

\section{Modèle de Données (Diagramme de Classes)}
La base de données relationnelle est structurée autour des utilisateurs, de leur historique et des scores calculés pour chaque smartphone.

\begin{figure}[h]
    \centering
    \includegraphics[width=0.8\textwidth]{rapport_resources/class.png}
    \caption{Diagramme de Classes}
\end{figure}

\section{Diagramme de Séquence : Analyse de Sentiment}
Ce diagramme détaille le flux d'information lors d'une requête d'analyse par l'utilisateur.

\begin{figure}[h]
    \centering
    \includegraphics[width=1.0\textwidth]{rapport_resources/sequence.png}
    \caption{Séquence d'Analyse}
\end{figure}

\chapter{Architecture Technique}

\section{Vue d'ensemble (Diagramme de Déploiement)}
L'application repose sur une architecture client-serveur classique.

\begin{figure}[h]
    \centering
    \includegraphics[width=0.8\textwidth]{rapport_resources/deployment.png}
    \caption{Architecture de Déploiement}
\end{figure}

\section{Choix Technologiques}
\subsection{Backend (Flask)}
Le choix de Python et Flask s'est imposé pour sa simplicité et son excellent support des bibliothèques de Data Science (NLTK). L'API expose des endpoints RESTful consommés par le frontend.
\subsection{Frontend (Angular)}
Angular (version 17+) a été choisi pour sa robustesse et son typage strict (TypeScript), garantissant une meilleure maintenabilité. L'interface utilise un "Design System" moderne (Soft UI).

\chapter{Conclusion}
Ce projet a permis de mettre en œuvre une chaîne complète de développement logiciel, de la conception UML à l'implémentation Full Stack. L'application résultante est fonctionnelle et répond aux objectifs initiaux d'analyse de réputation assistée par ordinateur.

\end{document}
